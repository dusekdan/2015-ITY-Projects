\documentclass[a4paper, 11pt]{article}
\usepackage[left=2cm,text={17cm, 24cm},top=4cm]{geometry}
\usepackage[utf8x]{inputenc}
\usepackage[czech]{babel}
\usepackage{times}
\pagestyle{empty}
\usepackage{url}
\DeclareUrlCommand\url{\def\UrlLeft{<}\def\UrlRight{>} \urlstyle{tt}}
\bibliographystyle{czplain}

\newcommand{\uv}[1]{\quotedblbase#1\textquotedblleft}

\begin{document}

\thispagestyle{empty}
\begin{center}
	\Huge
	\textsc{Vysoké učení technické v~Brně\\\vspace{-0.10cm}\huge{Fakulta informačních technologií}}\\
	\LARGE
	\vspace{\stretch{0.382}}
		
		\LARGE{Typografie a~publikování \,--\ 4. projekt}\\\vspace{-0.05cm}\Huge{Grafologie}
	
	\vspace{\stretch{0.618}}
	{\Large \today \hfill Daniel Dušek }
\end{center}

\newpage

\section*{Stručně o~grafologii v~dnešní společnosti}
\par Grafologie je vědecká disciplína zabývající se studií rysů lidského písma a~jeho možné provázanosti s~psychologickými rysy a~vlastnostmi daného člověka.
Toto označení pochází z~řeckého slova \uv{\textit{grapho}}, které v~doslovném překladu znamená \uv{psaní}. Autorem dnes používaného termínu Grafologie je~Francouz
J.~H.~Michon. \cite{MICHON}

\par Protože však dodnes všechny provedené výzkumy a~studie neprokázaly, ba naopak, některé i~vyvrátily, účinnost grafologie, je dnes považována vědeckou veřejností za~pseudovědu. \cite{DRIVER}

\par I~přes dnešní vědecký náhled na~grafologii se~však společnosti velmi často opírají o~poznatky této pseudovědy, což ji činí nepochybně zajímavým oborem. Na~základě posudku provedeného grafologickým specialistou
vyvozují pak závěry týkající se~zaměstnání uchazeče, \,-- například to, zda je uchazeč skrytě agresivní, nebo zachovává klidnou hlavu i~ve~stresových situacích.

\par Takovéto posudky lze dělat i~na~naprosto triviálních vzorcích písma \,-- například na~lidském podpisu. Schönfeld ve~své knize říká, že podpis je nejen jmennou pečetí, ale i~zkratkou osobnosti pro~vnější svět. Proto
zastáváme-li určité pozice ve~svém životě - pracovní, soukromou, můžou~se i~tyto dva podpisy lišit. Zejména u~osob se~sklonem k~schizofrenii pozorujeme odlišné podpisy, které jsou koreláty jednotlivých dílčích osobností. I~člověk nejskromnější je hrdý na~svoje jméno, proto je podpis dobrým ukazatelem sebecitu. Důležitý je také rozdíl mezi podpisem a~ostatním psaným projevem člověka. V podpisu se může do~krajnosti rozhýřit
sebecit člověka, který je navenek donucen ke skromnosti. \cite{SCHONFELD}

\par Navzdory všem diskutabilním stránkám grafologie ji i~dnes využívají některé státní orgány, jako je například policie, nebo BIS u~vyšetřování možných podvodů. Slovo grafologa má velikou váhu a~je na~něm možné budovat
i~případ, pokud vyjádří domněnku, že podpis mohl být zfalšován. Tento obor grafologie je nazývaný \textit{forenzní grafologie}. \cite{FORENSE}

\par Dále je dle grafologické filozofie možné z~písma rozeznat jak se člověk v~moment, kdy ho psal cítil. Toto lze dobře studovat například na denících. \cite{MASARNA2}

\par Dle optimistických názorových směrů lze grafologii využít k~zjištění a~včasnému vypozorování blížících se zdravotních poruch. \cite{HEALTH} Tato tvrzení ovšem nebyla vyvrácena, ale ani významným způsobem potvrzena.

\par Některé výzkumy se zabývají provázaností grafologie a~schopnosti kreslit. \cite{PAINT} Nepřekvapivě však z~nich vyplývá, že mezi těmito dvěmi činnostmi a~jejich viditelnými psychologickými znaky existuje pozitivní korelace. Výzkum žurnálu \textit{Experimentální psychologie} \cite{JOEP} navíc prokázal, že kreativní lidé v oblasti umění mívají velice často úhledné písmo, ne-li krasopis.

\par O provázanost grafologie a~kreativního vyžití se zajímala i~Mgr. Kamila~Gojná z~Brněnské Masarykovy Univerzity \cite{MASARNA}. Ve své práci došla k~velice podobným závěrům jako studie výše zmíněného žurnálu.

\par Podle stylu rukopisu jde také rozpoznat, zda text psal muž či žena, dítě, či dospělý, stařec či mládenec. Této studii se věnuje jeden dílčí článek výše již dvakrát spomenutého žurnálu. \cite{GP2}

\newpage
\bibliography{literatura}



\end{document}
